%----------------------------------------------------
% Setup Beamer
%----------------------------------------------------
\documentclass[hyperref={colorlinks=true}]{beamer}

%----------------------------------------------------
% Packages to use
%----------------------------------------------------
\input{../packages.sty}

%----------------------------------------------------
% Setup Theme
%----------------------------------------------------
\input{../theme.sty}

%----------------------------------------------------
% Table of Contents at each section transition
%----------------------------------------------------

\AtBeginSection[]
{
   \begin{frame}
       \frametitle{Outline}
       \setcounter{tocdepth}{2}
       \tableofcontents[currentsection]
   \end{frame}
}

%----------------------------------------------------
% Colors
%----------------------------------------------------
\input{../mycolors.sty}

%----------------------------------------------------
% Style, formatting, and new commands
%----------------------------------------------------
\input{../../global.sty}
\input{../newcommands.sty}
\input{../EandMcommands.sty}

%----------------------------------------------------
% Set paths for plots and images
%----------------------------------------------------
\input{../paths.sty}

%----------------------------------------------------
% SETTINGS FOR THIS LECTURE
%----------------------------------------------------
\newcommand{\lecnum }  {Lecture 9}
\newcommand{\lecdate}  {October 29, 2019}
\newcommand{\topic}    {Minimization, root finding, and solutions to ODEs}

%-----------------------------------------------------------------------------------------
% Title: [Column]{Title}
%-----------------------------------------------------------------------------------------
\title[PHYS 250 (Autumn 2019) -- \lecnum]{\topic}

%-----------------------------------------------------------------------------------------
% SubTitle: [Column]{Subtitle}
%-----------------------------------------------------------------------------------------
\subtitle{PHYS 250 (Autumn 2019) -- \lecnum}

%-----------------------------------------------------------------------------------------
% Author: [SubAuthor]{Author}
%-----------------------------------------------------------------------------------------
\author[D.W.~Miller]{David Miller}

%----------------------------------------------------
% Institute: [SubInst]{Institute}
%----------------------------------------------------
\institute[EFI, Chicago] 
{
  Department of Physics and the Enrico Fermi Institute\\
  University of Chicago
}

%----------------------------------------------------
% Institute: [SubInst]{Institute}
%----------------------------------------------------
\date[\lecdate]{\lecdate}

\subject{PHYS 250 Lecture}

\begin{document}

%==========================================================================================
% TITLE PAGE
%==========================================================================================

{
\begin{frame}
  \titlepage
\end{frame}
}

%==========================================================================================
\section[Connection between minimization and solving differential equations]{Connection between minimization and solving differential equations}
%==========================================================================================

%-----------------------------------------------------------------------------------------
\subsection[Reminders from Lecture 8]{Reminders from Lecture 8}
%-----------------------------------------------------------------------------------------

\begin{frame}%[shrink=10]
  \frametitle{Reminders from last time}

  There are many examples of optimization in physics
  
  \vspace{0.3cm}
  
  \begin{ucblock}{Examples of minimization}
    \begin{itemize}
      \item \bluebf{Fitting a model to data:} minimize differences between a model and data
      \item \bluebf{Second law of thermodynamics:} minimize changes in entropy for a system in thermodynamic equilibrium
      \item \bluebf{Conservation of momentum:} establish mechanical equilibrium by minimizing changes in momentum, $\frac{d\vec{p}}{dt}=0$
      \item \bluebf{Principle of least action:} obtain the equations of motion of a system by minimizing (or maximizing!) the variations of the action, $S$
      \item \bluebf{Path integral formulation of quantum mechanics:} sort quantum mechanically possible trajectories by minimizing quantum action
      \item \bluebf{Ising model:} minimization the energy of the spin configurations
    \end{itemize}
  \end{ucblock}

\end{frame}

%-----------------------------------------------------------------------------------------

\begin{frame}%[shrink=10]
  \frametitle{Optimization and root finding}

  Ultimately, I want to bring us to the point of discussing \bluebf{optimization in the context of modeling} (c.f. Lecture 1!). 
  
  \vspace{0.4cm}
  
  What that means, is that we want to have the computational tools to be able to \bluebf{model data and approximate complex functions computationally}, since it is often \alertbf{impossible to do so analytically}.
  
  \vspace{0.4cm}
  
  To achieve that, let's expand our tool kit beyond the analytically solvable example from last time (the 2D linear regression for $\chi^2$ minimization).

\end{frame}

%-----------------------------------------------------------------------------------------
\subsection[Categories of problem]{Categories of problem}
%-----------------------------------------------------------------------------------------

\begin{frame}%[shrink=20]
  \frametitle{Categorizing the problems}

  As we discussed in Lecture 8, an \bluebf{optimization problem} quickly turns into an attempt to solve a first-order ordinary (ODE) or partial differential equation (PDE) in order to identify the value of the parameter $\alpha$ for which the function $F$ is stationary:
  %
  \begin{equation}
    F^{\prime}(\alpha) = 0
  \end{equation}
  
  The approaches used to do this are closely related to the methods used for finding zeroes or \alertbf{``roots''} of functions (since the derivative of a function is, of course, also a function!). But we need to differentiate the problems:
  
  \begin{ucblock}{Initial vs. boundary value problems}
    Imagine you have the ODE
    %
    \begin{equation}
      F^{\prime\prime} = -F
    \end{equation}
    %
    \vspace{-0.5cm}
    %
    \begin{itemize}
      \item \bluebf{Initial value problem:}  $F(0)=1, F^{\prime}(0)=0$
      \item \bluebf{Boundary value problem:} $F(0)=1, F(\pi)=0$
    \end{itemize} 
    %
    The latter is often much harder!
  \end{ucblock}
  
  

\end{frame}

%-----------------------------------------------------------------------------------------
\subsection[Specific example]{Specific example}
%-----------------------------------------------------------------------------------------

\begin{frame}[shrink=20]
  \frametitle{Explicit root finding: square well potential}

  \begin{columns}
  
  \column{0.50\textwidth}
    
  The Schrodinger equation  
  %
  \begin{equation}
    \frac{\hbar^2}{2m}\psi^{\prime\prime}(x) + V(x)\psi(x) = E\psi(x)
  \end{equation}
  
  \pause
  
  tells us that for a finite square-well potential
  
  \begin{figure}
    \includegraphics[width=0.6\columnwidth]{Finite_Square_Potential_Well.png}
  \end{figure}
  
  \vspace{-0.5cm}
  
  \begin{equation}
    V(x) = \begin{cases}  
             -V_0 & 0 < x < L \\
             0    & \mathrm{else}
           \end{cases}
  \end{equation}
  
  \pause
  
  and a bound state ($E=-f<0$, $f$ positive definite) in the region $x<0$, we must solve:
  
  \begin{equation}
    -\frac{\hbar^2}{2m}\psi^{\prime\prime}(x<0) = -f\psi(x<0)
  \end{equation}
    
  \column{0.50\textwidth}

  \pause
  %
  The solution for $x<0$ is
  %
  \begin{equation}
    \psi(x<0) = A e^{\sqrt{z_0^2 - z^2}\, x}, \mathrm{where} \, z_0^2 - z^2 = \frac{2mf}{\hbar^2}
  \end{equation}
  
  and $z_0^2 \equiv 2mV_0/\hbar^2$.

  \pause

  In order to solve for the bound-state energies, we need to solve the following
  %
  \begin{equation}
    \tan(za) = \frac{2 \sqrt{ \left( \frac{z_0}{z} \right)^2 - 1 } }{ 2 - \left( \frac{z_0}{z} \right)^2 }
  \end{equation}
  
  \pause
  
  This is transcendental, requires numerical approaches to finding the \alertbf{``root''} of
  %
  \begin{equation}
    F(z) = \tan(za) - \frac{2 \sqrt{ \left( \frac{z_0}{z} \right)^2 - 1 } }{ 2 - \left( \frac{z_0}{z} \right)^2 }
  \end{equation}
  
  
  
  \end{columns}

\end{frame}

%==========================================================================================
\section[Finding roots]{Finding roots}
%==========================================================================================


%-----------------------------------------------------------------------------------------
\subsection[Bisection method]{Bisection method}
%-----------------------------------------------------------------------------------------

\begin{frame}%[shrink=10]
  \frametitle{Bisection method}

  The bisection method is straightforward and reliable.
  
  \mysp 
  
  Start with an interval defined by its endpoints, $x_{\ell}$ and $x_{r}$ on the left and right of a starting point, such that
  
  \begin{equation}
    F(x_{\ell})F(x_{r}) < 0
  \end{equation}
  
  That is, the function $F(x)$ has at least one root somewhere between $x_{\ell}$ and $x_{r}$. Then split that interval in half
  
  \begin{equation}
    x_m = \frac{1}{2}(x_{\ell} + x_{r})
  \end{equation}
  
  If the product $F(x_{\ell})F(x_{m}) < 0$, then a root of $F(x)$ lies between $x_{\ell}$ and $x_{m}$, otherwise, it lies between $x_{m}$ and $x_{r}$. Repeat until some tolerance $\epsilon$ is reached such that
  
  \begin{equation}
   | F(x_{m}) | \leq \epsilon
  \end{equation}
  
\end{frame}

%-----------------------------------------------------------------------------------------

\begin{frame}%[shrink=10]
  \frametitle{Bisection method}

  The bisection method is straightforward and reliable, \alertbf{but it can be slow if you don't have a good initial guess for the interval}.
  
  \mysp 
  
  If there is no root in the chosen interval, then it does not matter what you do.
  
  \mysp
  
  The method also uses no information whatsoever about the size of the step to take with each iteration. The overall time is determined \alertbf{only} by the size of the initial interval (i.e. how good your \alertbf{guess is}) and the tolerance, $\epsilon$.
  
  \mysp
  
  \pause
  
  \centering \bluebf{We can do better!}
  
\end{frame}

%==========================================================================================
%==========================================================================================
\end{document}
